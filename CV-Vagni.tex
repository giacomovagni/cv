
\documentclass[12pt,a4paper]{article}
\usepackage[german]{babel}
\usepackage[usenames, dvipsnames]{color}
\usepackage[none]{hyphenat}
\usepackage{hyperref}

\definecolor{mygray}{gray}{0.9}

\usepackage{currvita}
\newcommand*{\ac}[1]{\mbox{#1}}
\tolerance=600
\begin{document}
\begin{cv}{Curriculum Vitae}
  \begin{cvlist}{}
  \item \colorbox{mygray} {Giacomo Vagni}\\
  \emph{PhD Student in Sociology \textcolor{Gray}{(DPhil)}}
  
    \textcolor{Orange} {Centre for Time Use Research} \\
     Department of Sociology, \textcolor{Blue}{University of Oxford} 
    \item Nuffield College \\ 
    1 New Rd, Oxford OX1 1NF, United Kingdom
  \item Tel.:~(0044)~1865\,284464\\
    E"~Mail:~\href{mailto:giacomo.vagni@sociology.ox.ac.uk}{giacomo.vagni@sociology.ox.ac.uk}
    
    \item \href{https://giacomovagni.com} {https://giacomovagni.com} \\ 
     \href{https://www.sociology.ox.ac.uk/people/giacomo-vagni.html} {https://www.sociology.ox.ac.uk/people/giacomo-vagni.html} \\
    
  \end{cvlist}
  
  \begin{cvlist}{Research Interests}
  \item[] Time Use. Families. Social Stratification. \\  Sequence Analysis. Computational Social Science. \\ High Performance Computing. 
  \end{cvlist}
  
  \begin{cvlist}{Education}
  \item[2015--current] DPhil Student in Sociology. \\ University of Oxford. Nuffield College. UK. 
  \item[2012-2014] MA in Sociology. University of Geneva. CH. 
  \item[2008-2011] BA in Sociology \& Anthropology. \\ Universit{\'e} Libre de Bruxelles. Belgium. 
  \end{cvlist}
  
      \begin{cvlist}{Visiting}
      	\item[2017 (Aug-Dec)] Cornell University, Department of Sociology. USA 
      	\item[2017 (April)] National Institute for Demographic Studies (INED), France. 
      	\item[2014] Boston University, Department of Sociology. USA
      \end{cvlist}
      
 
 \newpage
 
   \begin{cvlist}{Publications}
	
		 \item [2018] Vagni, G. \& Cornwell, B. (2018). ``Patterns of everyday activities across social contexts.'' \emph{Proceedings of the National Academy of Sciences.} \href{https://doi.org/10.1073/pnas.1718020115}{doi: 10.1073/pnas.1718020115} 
		 
		   \item [] Bolzman, C., \& Vagni, G. (2018). ``And we are still here: Life courses and life conditions of Italian, Spanish and Portuguese retirees in Switzerland.'' In I. Vlase and B. Voicu (Eds.) \emph{Gender, Family, and Adaptation of Migrants in Europe.} (pp. 75-97). London:Palgrave Macmillan. \href{https://doi.org/10.1007/978-3-319-76657-7_4}{doi: 10.1007/978-3-319-76657-74} 
		    	 	 
   	 	  \item [] Vagni, G. (forthcoming). ``Family Time at the turn of the 21st century.'' In J. Gershuny and O. Sullivan (Eds.) \emph{The Great Day} (pp. 220-265). London:Penguin Books.  
   	 	    	 	    	 
   	\item [2016] Bolzman, C., \& Vagni, G. (2016). ``Forms of care among native Swiss and older migrants from Southern Europe: a comparison.'' \emph{Journal of Ethnic and Migration Studies}, 43(2), 250-269. 
   	\href{https://doi.org/10.1080/1369183X.2016.1238908}{doi: 10.1080/1369183X.2016.1238908} 

   	
   	\item [2015]
   	Bolzman, C., \& Vagni, G. (2015). ``{\'E}galit{\'e} des chances? Une
   	comparaison des conditions de vie des personnes {\^a}g{\'e}es immigr{\'e}es
   	et nationales.'' \emph{Hommes \& Migrations}, 1309, 19-28.
   \end{cvlist}
 
 
    \begin{cvlist}{Awards}
    	\item[2018]  Doc.Mobility (Swiss National Science Foundation)
    	\item[2017]  Brettschneider Exchange Fund (Cornell University)
    	\item [2015-2018] ERC Scholarship. University of Oxford. 
    	\item [2016]  Best Student Paper Award (Bronze). 38th International Association for Time Use Research. Seoul, Korea, July 22, 2016. 
    	\item [2014] Prize of Best Master Thesis - Sociology.  University of Geneva. 
    \end{cvlist}

  \begin{cvlist}{\ac{Expertise}}
  \item[Languages] R (\emph{expert}), Rcpp, Python (\emph{intermediate}),  C++ (\emph{beginner}) 
  
  \item[Statistics] Sequence Analysis (\emph{advanced}), 
  Clustering Analysis (\emph{intermediate}), 
  Ordinary Least Squares (\emph{intermediate}), Structural Equations Modelling (\emph{intermediate}), Multiple Correspondence Analysis (\emph{intermediate}), 
  Network Analysis (\emph{intermediate/beginner})
  
  \item[Tools] RStudio, Shiny, \ac{WEKA}, Ucinet, NetLogo, \ac{SPSS} \\ \LaTeX, Word, Excel, InDesign, \ac{VIM} \\ 
 \ac{HTML},  MySQL,  Github \\ 
 Experience in using ARCUS (Oxford Supercomputing facilities)
  \end{cvlist}

 \begin{cvlist}{Under Review}
 	
 	\item Vagni, G. (2017)  ``From Constraints to Culture. The Social Stratification of Parental Time.'' 
 	\item Vagni, G. (2017) ``Alone Together: Gender Inequalities in Couple Time''. 
 	
 \end{cvlist}
 
 \begin{cvlist}{Working Papers}
 	
 	\item Vagni, G., and Widmer, E. (2018)  ``Couple Time and Partnership Quality: an Empirical Assessment using Diary Data.''   \href{http://doi.org/10.17605/OSF.IO/K7NXM}{doi: 10.17605/OSF.IO/K7NXM} 
 	
 	\item Vagni, G. (2017) ``All You Need is Love. Searching for Causal Effects of Spouse on Enjoyment.'' 
 	
 	\item Vagni, G., and Sullivan, O. (2017)  ``Couple Work-schedules and the Division of Labor.''
 	
 \end{cvlist}

\begin{cvlist}{Affiliations and Academic Memberships}
	\item[Affiliate] Centre for Time Use Research (CTUR). Oxford; Nuffield College. Oxford; Sociological Research Institute (IRS). University of Geneva; Center for the Study of Inequality (CSI). Cornell University. 
	\item[Member] European Sociological Association (ESA); International Sociological Association (ISA); RC-28; The Royal Statistical Society (RSS); International Association of Time Use Research (IATUR); 
	British Society for Population Studies (BSPS). 
\end{cvlist}

\begin{cvlist}{Communications}
	
		\item[2017] ``From Constraints to Culture. The Social Stratification of Parental Time''. \emph{RC 28 Conference. International Sociological Association}, 8th of August. Columbia University, USA. 
				
	\item[] ``All You Need is Love. Searching for Causal Effects of Spouse on Enjoyment''. \emph{PopFest 2017}, 1-3 June. University of Stockholm, Sweden. 
	
	\item[] ``From Constraints to Culture.
The Social Stratification of Parental Time.''. \emph{DPhil Conference 2017}, 25th of May. University of Oxford, UK. 
	
	\item[]  ``From Constraints to Culture. Social Class Gradient in Early Parental Transmission of Cultural and Developmental Resources'', 4th of April. National Institute for Demographic Studies (INED), Paris, France. 
	
	\item[]  ``From Constraints to Culture. Social Class Gradient in Early Parental Transmission of Cultural and Developmental Resources''. \emph{ECSR Spring School Turin}, 13-17 March. Collegio Carlo Alberto, Italy. 
	
	 \item[2016]  ``Inequality and Family Time''. \emph{Time Use Workshop}, 29th of November. University of Tampere, Finland. 
	 
	 \item[]  ``Family time and Stratification''. \emph{Social Inequality Research Group}, 15th of November. Department of Social Policy, University of Oxford, UK.  
	 
	 \item[]  ``All You Need is Love. Searching for Causal Effects of Spouse on Enjoyment''. \emph{38th IATUR Conference}, 20-23 August. Seoul University, South Korea. 
	 
	 \item[]  ``Modelling Sequence Activities''. \emph{NCRM 7th ESRC Research Methods Festival}, 5-7 July. Bath University, UK. 
	 
	  \item[]  ``All You Need is Love. Searching for Causal Effects of Spouse on Enjoyment''. \emph{Graduate Research Seminar}, 6 June. Nuffield College, UK. 
	  
	   \item[2015]  ``Embodied Capital Accumulation and Employment Trajectories''. \emph{Time Use in Britain Conference}, 9-10 November. University of Oxford, UK.  
	   
	   	 \item[]  ``Time Together and Partnership Quality''. \emph{12th Conference of the European Sociological Association}, 25-28 August. Prague, Czech Republic.  
	   
	      \item[]  ``Exploring Co-Presence Data in the UK Time Use 2000''. \emph{37th IATUR Conference}, 5-7 August. Ankara, Turkey.  
	      
	       \item[2014]  ``Sharing Free Time''. \emph{36th IATUR Conference}, 5-7 August. University of Turku, Finland.  
	        
	        \item[2013] ``Un samedi dans la vie des couples en Suisse''. \emph{Research Seminar in Statistics for Social and Population Sciences}, University of Geneva, Switzerland. 
	        \end{cvlist}



  \cvplace{Oxford}
  \date{~May~2018}
\end{cv}
\end{document}
\endinput
%%
%% End of file `cvtest.tex'.
