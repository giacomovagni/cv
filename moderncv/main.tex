
\documentclass[11pt,a4paper,sans]{moderncv}        % possible options include font size ('10pt', '11pt' and '12pt'), paper size ('a4paper', 'letterpaper', 'a5paper', 'legalpaper', 'executivepaper' and 'landscape') and font family ('sans' and 'roman')

% moderncv themes
\moderncvstyle{classic}                            % style options are 'casual' (default), 'classic', 'oldstyle' and 'banking'
\moderncvcolor{green}                              % color options 'blue' (default), 'orange', 'green', 'red', 'purple', 'grey' and 'black'
%\renewcommand{\familydefault}{\sfdefault}         % to set the default font; use '\sfdefault' for the default sans serif font, '\rmdefault' for the default roman one, or any tex font name
%\nopagenumbers{}                                  % uncomment to suppress automatic page numbering for CVs longer than one page

% character encoding
\usepackage[utf8]{inputenc}                       % if you are not using xelatex ou lualatex, replace by the encoding you are using
%\usepackage{CJKutf8}                              % if you need to use CJK to typeset your resume in Chinese, Japanese or Korean

% adjust the page margins
\usepackage[scale=0.75]{geometry}
%\setlength{\hintscolumnwidth}{3cm}                % if you want to change the width of the column with the dates
%\setlength{\makecvtitlenamewidth}{10cm}           % for the 'classic' style, if you want to force the width allocated to your name and avoid line breaks. be careful though, the length is normally calculated to avoid any overlap with your personal info; use this at your own typographical risks...

% personal data
\name{Giacomo}{Vagni}
\title{Resumé title}                               % optional, remove / comment the line if not wanted
\address{DPhil Student in Sociology}{University of Oxford}{Nuffield College}% optional, remove / comment the line if not wanted; the "postcode city" and and "country" arguments can be omitted or provided empty
% optional,  \phone[mobile]{+1~(234)~567~890}                   
\email{giacomo.vagni@nuffield.ox.ac.uk }                               % optional, remove / comment the line if not wanted
\homepage{giacomovagni.github.io/personal/}                         % optional, remove / comment the line if not wanted
% optional, remove / \extrainfo{additional information}                  
%\photo[64pt][0.4pt]{picture}                       % optional, remove / comment the line if not wanted; '64pt' is the height the picture must be resized to, 0.4pt is the thickness of the frame around it (put it to 0pt for no frame) and 'picture' is the name of the picture file
\quote{Some quote}                                 % optional, remove / comment the line if not wanted


\begin{document}
%-----       letter       ---------------------------------------------------------
% recipient data
\recipient{DPIR}{University of Oxford}
\date{31 May 2017}
\opening{To whom it may concern,}
\closing{Yours faithfully,}
\enclosure[Attached]{curriculum vit\ae{}}          % use an optional argument to use a string other than "Enclosure", or redefine \enclname
% optional, remove /
\makelettertitle

I am writing to apply for the “Quantitative Methods Graduate Teaching Assistantships” position. 

I am a second year DPhil student in Sociology. My work focuses on the social stratification of parental time in the UK using time use diaries. I have a strong programming background in R (\texttt{dplyr, data.table, Rcpp}). I also have some programming experience in \texttt{C++} and \texttt{Python}. 

I am very interested in statistical programming, visual statistics and high performance computing. I have been involved in several projects involving ``big data''. 
Another research interest of mine is reproducible research and in particular reproducible statistical analysis. I am familiar with \texttt{Knitr} (\texttt{RSweave} and \texttt{RMarkdown}), Github and in writing and sharing R packages.

 I have a basic knowledge of regression analysis (OLS, multilevel and fixed effects) but I am more interested in computer science based methods such as sequence analysis. I have great interest in visual and exploratory statistics and I have an extensive knowledge of \texttt{ggplot2}. 

I think that I am an ideal candidate because of my interest in programming and reproducible research. Firstly, I am qualified to teach R basics and R programming. Secondly, I can teach students how to create powerful workflows and reproducible analysis using \texttt{Knitr} and Github. 

Supervisor details: Professor Oriel Sullivan, \href{mailto:oriel.sullivan@sociology.ox.ac.uk}{oriel.sullivan@sociology.ox.ac.uk}, St Hugh's College.


\makeletterclosing

\end{document}


%% end of file `template.tex'.
